
\documentclass[12pt]{article}

\setlength{\oddsidemargin}{0pt}
\setlength{\textwidth}{470pt}
\usepackage{amssymb}
\usepackage{amsfonts}
\usepackage{amsmath}
\usepackage{latexsym}
\usepackage{epsfig}

\newtheorem{prop}{Proposition}
\newenvironment{proof}{\noindent \textbf{Proof:}}{$\Box$}
\newcommand{\infint}{\int_{-\infty}^\infty}
\newcommand{\intunit}{\int_{-1}^1}

\title{CS Latex Problem Set \#1}

\author{Xxxx Y. Zzzzzzzzz\\
{\tt xyz1234@fit.hcmus.edu.vn}
}

\begin{document}

\maketitle

\section{Problem 1}
 
Insert a correct solution to problem 1 here.

\section{Problem 2}

Insert a correct solution to problem 2 here.  And so on.  

\medskip

The two most basic things to know about \LaTeX are:

\begin{itemize}
\item Extra         spaces     in     the     input     file
don't
matter
(compare this output with what's in the .tex file).  However, if you have a blank
line in the .tex file, this makes whatever comes after it come on a new line

like this.

\item To put something in math mode, surround it with dollar signs.  If you do
this correctly you will get nice-looking math like $x + y = z$.  If you do it incorrectly
you'll get an error message; if you accidentally have non-math-mode text contained inside
dollar signs $it will come out looking very bad.$

\end{itemize}

\LaTeX is very good at handling mathematical symbols, equations, and so on.  To give you
a sense of what it can do, 
here's a sample of some math stuff.  
By comparing the output with the stuff in the 
.tex file, you should be able to see how to use most of the basic features in 
\LaTeX.  

\bigskip
\bigskip

\begin{prop}
For $1 \leq k \leq n$ let $X_k$ be the random variable which is $-1$ with probability 
$p_k$ and $+1$ with probability $1-p_k$\@.  Let $x = \sum_{k=1}^n X_k$.  Then for every $\theta$,
\[\Pr[|x-\theta| \leq 1] \leq \frac{O(1)}{\sqrt{\sum_{k=1}^n p_k(1-p_k)}}.\]
\end{prop}

\begin{proof}
Define:
\[p(x) = \frac{2(1-\cos{x})}{x^2} \geq 0\qquad\text{and}\qquad h(t) = \begin{cases} 1-|t|, & |t| 
\leq 1 \\ 0, & \text{else} \end{cases}.\]

\noindent Elementary integration by parts shows that $p(x)$ is the inverse Fourier transform of 
$h(t)$; i.e.,
\[p(x) = \infint e^{-itx} h(t)\,dt.\]

\noindent By considering the Taylor expansion of $\cos{x}$, we get that $p(x) \geq \frac{11}{12}$ 
on the interval $[-1,1]$.  Hence:
\newcommand{\twel}{\frac{12}{11}}
\begin{eqnarray}
\Pr[|x-\theta| \leq 1] & = & E_x[\mathbf{1}_{x \in [\theta-1,\theta+1]}] \nonumber \\
& \leq & {\textstyle \twel} E[p(x-\theta)] \nonumber \\
& = & {\textstyle \twel} E\Bigl[\infint e^{-it(x-\theta)}h(t)\,dt\Bigr] \nonumber \\
& = & \twel \infint E[e^{-itx} e^{it\theta} h(t)]\,dt \nonumber \\
& = & \twel \Bigl|\infint e^{it\theta} h(t) E[e^{-itx}]\,dt\Bigr| \label{eqn:1}\\
& \leq & \twel \intunit \bigl|E[e^{-itx}]\bigr|\,dt, \label{eqn:2}
\end{eqnarray}
with (\ref{eqn:1}) following because the quantity is already real and nonnegative, and 
(\ref{eqn:2}) following because $|e^{it\theta}| \leq 1$, $h(t) = 0$ outside $[-1,1]$, and $|h(t)| 
\leq 1$ otherwise.\\

\noindent Now,
\begin{eqnarray*}
E_x[e^{-itx}] & = & E_{x_k \leftarrow X_k} [\exp (-it\sum_{k=1}^n x_k\bigr)] \\
& = & E_{x_k \leftarrow X_k} \Bigl[\prod_{k=1}^n \exp (-itx_k)\Bigr] \\
& = & \prod_{k=1}^n E_{x_k \leftarrow X_k}[\exp(-itx_k)] \qquad \text{(by independence)}\\
& = & \prod_{k=1}^n (p_k \exp(it) + (1-p_k) \exp(-it)) \\
& = & \prod_{k=1}^n (\cos t + i(2p_k-1)\sin t).
\end{eqnarray*}

\noindent By comparing Taylor expansions one can establish that
$$
|\cos t + i(2p-1) \sin t| \leq \exp(-{\frac {11}{24}}p(1-p)t^2).
$$
for $p \in [0,1], t\in[-1,1].$
We may conclude that:
\begin{eqnarray*}
\Pr[|x-\theta| \leq 1] & \leq & \twel \intunit \prod_{k=1}^n \exp(-{\textstyle 
\frac{11}{24}}p_k(1-p_k)t^2) \,dt \\
& = & \twel \intunit \exp(-{\textstyle \frac{11}{24}}\left[\sum_{k=1}^n p_k(1-p_k)\right]t^2) \,dt 
\\
& = & \twel \intunit \exp(-t^2/2({\scriptstyle \sqrt{\twel}}\sqrt{\sum p_k(1-p_k)}^{-1})^2)\,dt \\
& \leq & \twel \infint \exp(-t^2/2({\scriptstyle \sqrt{\twel}}\sqrt{\sum p_k(1-p_k)}^{-1})^2)\,dt 
\\
& = & \sqrt{2\pi}({\textstyle \twel})^{3/2} \sqrt{\sum p_k(1-p_k)}^{-1},
\end{eqnarray*}
since $(\sqrt{2\pi} \sigma)^{-1} \exp(-t^2/2\sigma^2)$ is a probability density function for every 
positive $\sigma$.
\end{proof}

\end{document}




